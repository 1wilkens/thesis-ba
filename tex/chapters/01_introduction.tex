\chapter{Introduction}
\label{chap:Introduction}

\abstract{%
This chapter aims to provide some background information to high-performance computing. The first section describes problems with the currently used programming languages and motivates the search for new candidates. After that the chapter concludes with a quick rundown of the thesis' goals.
}

\section{Motivation}
\label{sec:Motivation}

The world of high-performance computing is evolving rapidly and programming languages used in this environment are held up to a very high standard. It comes as no surprise that runtime performance is the top priority in language selection when an hour of computation costs thousands of dollars. The focus on raw power led to C and Fortran having an almost monopolistic position in the industry, because their execution speed is nearly unmatched.
\\ \\
However programming in these rather antique languages can be rather difficult. Although they are still in active development, their long lifespans resulted in sometimes unintuitive syntax and inconsistent behaviour. Especially C's \textit{undefined behaviour} often causes inexperienced programmers to write unreliable code which is unnecessarily dependant on implementation details of a specific compiler. Understanding and maintaining these programs requires deep knowledge of memory layout and other computer internals?
\\ \\
Considering the fact that scientific applications are mostly written by scientist without a concrete background in information technology it is evident that the current situation is less than ideal.
- new technologies emerged
    - badly (and late) or not supported at all by c and fortran
    -> time to evaluate new languages while keeping performance in mind

\section{Goals of this thesis}
\label{sec:Goals}

This thesis aim to evaluate Rust and Go as potential programming languages in the high-performance computing environment. The comparison is based on a reimplementation of an existing parallel application in the two languages as well as C. This application is streets4MPI, a traffic simulation software written in Python using MPI to parellelize calculations. Since libraries for interprocess communication in Rust and Go are nowhere near production-ready this thesis will focus on shared memory parallelization to avoid unfair bias based solely on the quality of the supporting ecosystem.
\\ \\
The final application is a simplified version of the original streets4MPI but will behave nearly identical. It uses the OpenStreeMap Project's .osm.pbf files as input and writes the results to a custom output format for later analysis. To reduce complexity it does not support additional commandline arguments and has limited error handling regarding in- and output.
\\ \\
While performance will be the main concern additional software metrics will also be reviewed to measure the complexity and overall quality of the produced applications. Another aspect to review is the tool support and ease of development.

- evaluate languages for use in (scientific) high-performance computing
    - shared memory -> thread-parallelization
    - tools, (common) frameworks,
- through implementation of streets4mpi in C, go and rust
    - performance(!)
    - metrics
