\chapter{Introduction}
\label{chap:Introduction}

\textit{%
In diesem Kapitel ...
}
\bigskip

\section{Motivation}
\label{sec:Motivation}

The world of high-performance computing is evolving rapidly and programming languages used in this environment are held up to a very high standard. It comes as no surprise that runtime performance is the top priority in language selection when an hour of computation costs thousands of dollars. The focus on raw power led to C and Fortran having an almost monopolistic position in the industry, because their execution speed is nearly unmatched.
\\ \\
However programming in these rather antique languages can be rather difficult. Although they are still in active development, their long lifespans resulted in sometimes unintuitive syntax and inconsistent behaviour. Especially C's \textit{undefined behaviour} often causes inexperienced programmers to write unreliable code which is unnecessarily dependant on implementation details of a specific compiler. Understanding and maintaining these programs requires deep knowledge of memory layout and other computer internals?
\\ \\
Considering the fact that scientific applications are mostly written by scientist without a concrete background in information technology it is evident that the current situation is less than ideal.
- new technologies emerged
    - badly (and late) or not supported at all by c and fortran
    -> time to evaluate new languages while kepping performance in mind

\section{Goals of this Thesis}
\label{sec:Goals}

- evaluate languages for use in (scientific) high-performance computing
    - shared memory -> thread-parallelization
    - tools, (common) frameworks,
- through implementation of streets4mpi in C, go and rust
    - performance(!)
    - metrics
