\chapter*{Abstract}
\label{ch:Abstract}

\thispagestyle{empty}

This thesis aims to analyze new programming languages in the context of \gls{hpc}. In contrast to many other evaluations the focus is not only on \textbf{performance} but also on developer \textbf{productivity metrics}. The two new languages Go and Rust are compared with C as it is one of the two commonly used languages in \gls{hpc} next to Fortran.

The base for the evaluation is a shortest path calculation based on real world geographical data which is parallelized for shared memory concurrency. An implementation of this concept was written in all three languages to compare multiple productivity and performance metrics like \textbf{execution time}, \textbf{tooling support}, \textbf{memory consumption} and \textbf{development time} across different phases.

Although the results are not comprehensive enough to invalidate C as a leading language in \gls{hpc} they clearly show that both Rust and Go offer tremendous \textbf{productivity gain} compared to C with \textbf{similar performance}. Additional work is required to further validate these results as future reseach topics are listed at the end of the thesis.
