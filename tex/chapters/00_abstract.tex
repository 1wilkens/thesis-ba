\chapter*{Abstract}

\thispagestyle{empty}

The world of high-performance computing is evolving rapidly and programming languages used in
this environment are held up to a very high standard. It comes as no surprise that runtime performance
is the top priority in language selection when an hour of computation costs thousands of dollars.
The focus on raw power led to C and Fortran having an almost monopolistic position in the industry,
because their execution speed is nearly unmatched.
\\ \\
This thesis aims to analyze new programming languages in the context of HPC. To compare not only speed
but also development productivity and general inner metrics, a basic traffic simulation is implemented
in C, Mozilla's Rust and Google's Go. These two languages were chosen on their basic promise of
performance as well as memory-safety in the case of Rust or easy multhithreaded execution (Go).
The implementations are limited to shared-memory parallelism to achieve a fair comparison since the
library support for inter-process communication is rather limited at the moment.
\\
Nonetheless the comparison should allow a decent rating of the viability of these two languages in
high-performance computing.

% Really use company names?
% Last sentence is not symmetrical between languages
