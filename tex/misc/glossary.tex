% acronyms
\newacronym{hpc}{HPC}{high-performance computing}
\newacronym{sloc}{SLOC}{source lines of code}
\newacronym{tdd}{TDD}{test-driven development}
\newacronym{xml}{XML}{Extensible Markup Language}

% regular terms
\newglossaryentry{goroutine}{
    name={goroutine},
    description={A lightweigth concurrently executing function which gets multiplexed into OS threads by the Go runtime~\cite{effective_go}},
    plural={goroutines}
}
\newglossaryentry{llvm}{
    name={LLVM},
    description={The LLVM Compiler Infrastructure Project (formerly short for Low Level Virtual Machine) is an umbrella project for various compiler and other low-level tools. LLVM Core is the primary subproject and a set of libraries for code generation and optimization for various platforms.~\cite{llvm_main}}
}
\newglossaryentry{protobuf}{
    name={Protocol Buffers},
    description={``Protocol buffers are Google's language-neutral, platform-neutral, extensible mechanism for serializing structured data''~\cite{protobuf}}
}

% dual entries
\newdualentry
{beam}
{BEAM}
{Bogdan/Bj\"orn's Erlang Abstract Machine}
{The virtual machine which runs Erlang. It loads bytecode which is converted directly to threaded native code and executed.}
\newdualentry
{dkrz}
{DKRZ}
{Deutsches Klimarechenzentrum}
{The German Climate Computing Centre (german: Deutsches Klimarechenzentrum) ``is a central service center for the German climate research and Earth system research. It operates high performance computing for applied and basic research in climate science and related disciplines.''~\cite{wiki_dkrz}}
\newdualentry
{llvm_ir}
{LLVM IR}
{LLVM Intermediate Representation}
{A low level programming language similar to assembly. It is the code representation LLVM uses in its Core libraries. LLVM IR is platform-agnostic with the ``capability of representing `all` high-level languages cleanly.''~\cite{llvm_ir_ref}}
\newdualentry
{mpi}
{MPI}
{Message Passing Interface}
{The Message Passing Interface standard is a library specification developed by a committee of vendors, implementors and users. It is the current cominant model used in high-performance computing and implementations for many platforms (both commercial and free) are available including bindings for various programming languages.~\cite{mpi_main, mpi_infiniband}}
\newdualentry
{osm}
{OSM}
{OpenStreetMap}
{OpenStreetMap is a collaborative project which aims collect and maintain geographical data about roads, trails, railways stations and more. As the name suggests the data is provided openly under the Open Data Commons Open Database License~\fnote{\url{http://opendatacommons.org/licenses/odbl/}}}
\newdualentry
{sssp}
{SSSP}
{Single Source Shortest Path}
{A commong graph problem searching for shortest paths between nodes of a graph. As the name suggests this type of problem states the use a single node as starting point and aims to determine shortest paths to all remaining nodes of the graph. Dijkstra's algorithm is commonly used for graphs with nonnegative edge weights while the Bellman-Ford-Algorithm can even handle that case.}
