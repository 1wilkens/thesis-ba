\documentclass[
	12pt,
	a4paper,
	BCOR10mm,
	%chapterprefix,
	DIV14,
	listof=totoc,
	bibliography=totoc,
	headsepline
]{scrreprt}

% Encoding Options
\usepackage[T1]{fontenc}
\usepackage[utf8]{inputenc}
% Language
\usepackage[english]{babel}

% BibLaTex incl. Biber as backend
\usepackage[
	% General options
	backend=biber,
	style=alphabetic,
	sortlocale=en_US,
	% Specific options
	% backref=true,
	doi=true,
	eprint=false,
	hyperref=true,
	url=true]{biblatex}
\addbibresource{sources.bib}

% change date format from 'mm/dd/yyyy' to 'dd. mm . yyyy' TODO: revert later?
\renewbibmacro*{url+urldate}{%
\printfield{url}%
\iffieldundef{urlyear}
	{}
	{\setunit*{\addspace}%
	\printtext[urldate]{\printfield{urlday}\setunit*{\adddot\addthinspace}%
			\printfield{urlmonth}\setunit*{\adddot\addthinspace}%
			\printfield{urlyear}%\setunit*{\adddot}
}}}%

\addto\captionsenglish{% Replace "english" with the language you use
	\renewcommand{\contentsname}{Table of Contents}
}

% Modern latin font
\usepackage{lmodern}

% Load xcolor first
\usepackage[usenames, dvipsnames]{xcolor}

% Various useful packages
\usepackage[footnote]{acronym}
\usepackage[page]{appendix}
\usepackage{booktabs}
\usepackage{caption}
\usepackage[autostyle,babel,strict]{csquotes}
\usepackage{footnote}
\usepackage{float}
\usepackage{graphicx}
\usepackage[htt]{hyphenat}
\usepackage{listings}
\usepackage{microtype}
\usepackage{pdflscape}
\usepackage{pgfplots}
\usepackage{scrlayer-scrpage}
\usepackage{subfig}
\usepackage{textcomp}
\usepackage{tikz}
\usepackage{tikzscale}
\usepackage[subfigure,titles]{tocloft}
\usepackage{units}
\usepackage{xparse}

% For dates
\newcommand{\leadingzero}[1]{\ifnum #1<10 0\the#1\else\the#1\fi}

% Allow footnotes in tables
\makesavenoteenv{tabular}
\makesavenoteenv{table}

% Use hyperref last to make sure it can define all necessary commands
\usepackage[pdfborder={0 0 0}]{hyperref}

\addto\extrasenglish{%
	\def\chapterautorefname{Chapter}%
	\def\sectionautorefname{Section}%
}

% Use glossaries/xindy after hyperref to make it use links
\usepackage[nonumberlist,nopostdot,numberedsection=autolabel,style=altlist,toc,xindy]{glossaries}
\usepackage[xindy]{imakeidx}

% Command to create dual entries (acronyms with description)
\DeclareDocumentCommand{\newdualentry}{ O{} O{} m m m m } {
  \newglossaryentry{gls-#3}{name={#5},text={#5\glsadd{#3}},
    description={#6},#1
  }
  \newacronym[see={[Glossary:]{gls-#3}},#2]{#3}{#4}{#5\glsadd{gls-#3}}
}

% Setup glossaries
\loadglsentries[main]{misc/glossary}
\makeglossaries
\makeindex
% \renewcommand*{\glsautoprefix}{ap:gl_}

% Limit ToC depth
\setcounter{tocdepth}{1}

% Page (indent) definitions
\sloppy
\setlength{\parindent}{0em}
\setlength{\parskip}{1.2ex plus 0.5ex minus 0.5ex}

% Some pdf related meta-information
\hypersetup{
	pdftitle	={Evaluation of performance and productivity metrics of potential programming languages in the HPC environment},
	pdfauthor 	={Florian Wilkens},
	pdfkeywords ={HPC, Rust, C, Go, performance, productivity}
}

% Set default graphicspath to 'img' folder
\graphicspath{img}

% Define some custom colors
\definecolor{light-gray}{gray}{0.85}
\definecolor{comment-green}{HTML}{009900}

% Define language environments or listings
\lstdefinestyle{base}{
	basicstyle=\ttfamily,
	breakatwhitespace=true,
	breaklines=true,
	captionpos=b,
	commentstyle=\color{comment-green},
	frame=single,
	keywordstyle=\color{blue},
	numbers=left,
	numberstyle=\tiny,
	rulecolor=\color{black},
	% prebreak=\textbackslash,
	postbreak=\hbox{$\hookrightarrow$ },
	showstringspaces=false,
	stringstyle=\color{red},
	tabsize=4
}

\lstdefinestyle{plain}{
	style=base,
	numbers=none
}

\lstdefinestyle{shell}{
	style=plain
	%language=sh
}

\lstdefinestyle{c}{
	language=C,
	style=base,
	morekeywords={pragma, omp}
}

\lstdefinestyle{go}{
	language=C,
	style=base,
	morekeywords={func, type, import, go, make, package, uint, int8, uint8, int16, uint16, int32, uint32, int64, uint64, float32, float64}
}

\lstdefinestyle{rust}{
	language=C,
	style=base,
	morekeywords={fn, let, self, match, move, mut, in, use, i8, u8, i16, u16, i32, u32, i64, u64, isize, usize, f32, f64, Vec},
	stringstyle=\color{black}
}

\lstdefinestyle{python}{
	language=python,
	style=base
}

\lstdefinestyle{erlang}{
	language=erlang,
	style=base,
	morekeywords={MODULE}
}

% Define some custom commands for inlining code, file names or shell commands
\newcommand{\mdinline}[1]{\colorbox{light-gray}{\texttt{#1}}}
\newcommand{\shinline}[1]{\textbf{\texttt{#1}}}

% Custom footnote command that adds \hspace before the text
\newcommand{\fnote}[1]{\footnote{\hspace{2pt}#1}}

% Change name of listing from 'Listings' to 'List of Listings' to be consistent
\renewcommand*{\lstlistlistingname}{List of Listings}

% Change abstract command to something useful
\renewcommand{\abstract}[1]{\textit{#1}\bigskip}

\begin{document}

\begin{titlepage}
    \begin{center}
        {\titlefont\huge Evaluation of performance and productivity metrics of
            potential programming languages in the HPC environment \par}

        \bigskip
        \bigskip

        {\titlefont\Large --- Bachelor Thesis ---\par}

        \bigskip
        \bigskip

        {\large Division Scientific Computing \\
        Department of Informatics \\
        Faculty of Mathematics, Informatics und Natural Sciences \\
        University of Hamburg\par}
    \end{center}

    \vfill

    {\large \begin{tabular}{ll}
        Submitted by: & Florian Wilkens \\
        E-Mail: & \href{mailto:1wilkens@informatik.uni-hamburg.de}
            {1wilkens@informatik.uni-hamburg.de} \\
        Matriculation number: & 6324030 \\
        Course of studies: & Software-System-Entwicklung \\
        \\
        First assessor: & Prof. Dr. Thomas Ludwig \\
        Second assessor: & Sandra Schr\"oder \\ \\
        Advisor: & Michael Kuhn, Sandra Schr\"oder \\
        \\
        Hamburg, \today
    \end{tabular}\par}
\end{titlepage}


\chapter*{Abstract}
\label{ch:Abstract}

\thispagestyle{empty}

This thesis aims to analyze new programming languages in the context of \gls{hpc}. In contrast to many other evaluations the focus is not only on \textbf{performance} but also on developer \textbf{productivity metrics}. The two new languages Go and Rust are compared with C as it is one of the two commonly used languages in \gls{hpc} next to Fortran.

The base for the evaluation is a shortest path calculation based on real world geographical data which is parallelized for shared memory concurrency. An implementation of this concept was written in all three languages to compare multiple productivity and performance metrics like \textbf{execution time}, \textbf{tooling support}, \textbf{memory consumption} and \textbf{development time} across different phases.

Although the results are not comprehensive enough to invalidate C as a leading language in \gls{hpc} they clearly show that both Rust and Go offer tremendous \textbf{productivity gain} compared to C with \textbf{similar performance}. Additional work is required to further validate these results as future reseach topics are listed at the end of the thesis.


\microtypesetup{protrusion=false} % disable protrusion locally for the toc
\tableofcontents 				  % prints Table of Contents
\microtypesetup{protrusion=true}  % reenable protrusion

\chapter{Introduction}
\label{chap:Introduction}

\abstract{%
This chapter provides some background information to high-performance computing. The first section describes problems with the currently used programming languages and motivates the search for new candidates. After that the chapter concludes with a quick rundown of the thesis' goals.
}

\section{Motivation}
\label{sec:Introduction::Motivation}

The world of high-performance computing is evolving rapidly and programming languages used in this environment are held up to a very high standard. It comes as no surprise that runtime performance is the top priority in language selection when an hour of computation costs thousands of dollars. The focus on raw power led to C and Fortran having an almost monopolistic position in the industry, because their execution speed is nearly unmatched.

However programming in these rather antique languages can be rather difficult. Although they are still in active development, their long lifespans resulted in sometimes unintuitive syntax and large amounts of historical debt. Especially C's \textit{undefined behaviour} often causes inexperienced programmers to write unreliable code which is unnecessarily dependant on implementation details of a specific compiler. Understanding and maintaining these programs requires deep knowledge of memory layout and other technical details.

Considering the fact that scientific applications are often written by scientist without a concrete background in compuer science it is evident that the current situation is less than ideal. There have been various efforts to make programming languages more accessible in the recent years but unfortunately none of the newly emerged ones have been successful in establishing themselves in the HPC community to this day. Although many features and concepts have found their way in newer revision of C and Fortran standards most of them feel tacked (//wording) on and are not well integrated into the core languages.

One example for this is the common practice of testing. Specifically with the growing popularity of \textif{test driven development} it became vital to the development process to be able to quickly and regularly execute a set of tests to verfiy the ongoing work. Of course there are also testing frameworks and libraries for Fortran and C but since these languages lack deep integration of testing concepts, they often require a lot of setup and boilerplate code and are generelly not that pleasant to work with. In contrast for example the Go programming language includes a complete testing framework with the ability to perform benchmarks, execute global setup/teardown work and even do basic output testing.~\cite{GO_TEST} (// cite vs footnote) Maybe most important all this is available via a single executable \lstinline$go test$ which may be easily integrated in scripts or other parts of the workflow.

While testing is just one example there are a lot of ``best practices'' and techniques which can greatly increase both developer productivity and code quality but require a language-level integration to work best. Combined with the advancements in type system theory and compiler construction both C and Fortran's feature sets look very dated. With this in mind it is time to evaluate new potential successors of the two giants of high-performance computing.

\section{Goals of this thesis}
\label{sec:Introduction::Goals}

This thesis aims to evaluate Rust and Go as potential programming languages in the high-performance computing environment. The comparison is based on a reimplementation of an existing parallel application in the two languages as well as C. This application is streets4MPI, a traffic simulation software written in Python using MPI to parellelize calculations. Since libraries for interprocess communication in Rust and Go are nowhere near production-ready this thesis will focus on shared memory parallelization to avoid unfair bias based solely on the quality of the supporting library ecosystem.

The final application is a simplified version of the original streets4MPI but will behave nearly identical. It uses the OpenStreeMap Project's .osm.pbf files as input and writes the results to a custom output format for later analysis. To reduce complexity it does not support additional commandline arguments and has limited error handling regarding in- and output.

While performance will be the main concern additional software metrics will also be reviewed to measure the complexity and overall quality of the produced applications. Another aspect to review is the tool support and ease of development.

% - evaluate languages for use in (scientific) high-performance computing
    % - shared memory -> thread-parallelization
    % - tools, (common) frameworks,
% - through implementation of streets4mpi in C, go and rust
    % - performance(!)
    % - metrics


\chapter{State of the art}
\label{chap:State_of_the_art}

\abstract{%
This chapter describes the current state of the art in high-performance computing. The dominance of Fortran and C is explained in~\autoref{sec:State_of_the_art::Weaknesses_C_Fortran} and after that all considered language candidates are introduced and characterized.
}

- state of C and Fortran (section name?)

- technological advancements in low level languages
    - static analysis
    - ..
    -> But no real adaption possible, because language level support is missing (already included in introduction)

\section{Weaknesses of C and Fortran} % change name?
\label{sec:State_of_the_art::Weaknesses}

As stated in~\autoref{sec:Introduction::Motivation} high-performance computing is largely dominated by C and Fortran. To understand why a new language is needed it is essential to understand the shortcomings of these programming veterans. (//language?)

\subsection*{}

// Candidates here for now might need another chapter for those
\section{Candidates}
\label{sec:State_of_the_art::Candidates}

This section aims to provide a rough overview of possible candidates that were considered for further evaluation in this thesis.

\subsection*{Python}
\label{subsec:State_of_the_art::Candidates::Python}

Python is an interpreted general-purpose programming language which aims to be very expressive and flexible. Compared with C and Fortran which sacrifice feature richness for performance, Python's huge standard library combined with the automatic memory management offers a low border of entry and quick prototyping capabilities.

%  wording
As a matter of fact many introductory computer science courses at universities in the United States recently switched from Java to Python as their first programing language.~\cite{GUO14, intro_py} This allows the students to focus on core concepts and algorithms instead of boilerplate code.
\\
\lstinputlisting[caption={FizzBuzz in Pyhon 3.4}, label={lst:example.py}, style=python]{code/example.py}

In addition to the very extensive standard library the Python community has created a lot of open source projects aiming to support especially scientific applications. There is NumPy~\footnote{\url{http://www.numpy.org}} which offers efficient implementations for multidimensional arrays and common numeric algorithms like Fourier transforms or MPI4Py~\footnote{\url{http://www.mpi4py.scipy.org}}, an MPI abstraction layer able to interface with various backends like OpenMPI or MPICH. Especially the existance of the latter shows the ongoing attempts to use Python in a cluster environment and there have been successful examples of scientific high-performance applications using these libraries( //need ref ).

Unfortunately dynamic typing and automatic memory management come at a rather high price. The speed of raw numeric algorithms written in plain Python is almost always orders of magnitude slower than implementations in C or Fortran. As a consequence nearly all of the mentioned libraries implement the critical routines in C and focus in optimizing the interop (// wording) experience. This often means one needs to make tradeoffs between idiomatic Python - which might not be transferable to the foreign language - and maximum performance. As a result performance critical python code often looks like it's equivalent written in a statically typed language and the more terseness Python loses because of this the less desireable it becomes to use in high-performance computing since one could just fall back to C for a similar experience.

In conclusion Python was not chosen to be further evaluated because of the mentioned lack of performance (in pure Python). This might change with some new implementations emerging recently though. Most of the problems discussed here are present in all stable Python implementations today (most notably \textit{Cython} and \textit{PyPy}) but new projects aim to improve the execution speed in various ways. \textit{Medusa} compiles Python code to Google's Dart to make use of the underlying virtual machine. Although these ventures are still in early phases of development, first early benchmarks promise drastic performance improvements. Once Python can achieve similar execution speed to native code (by a maximum of one order of magnitude) it will become a serious competitor in the high performance area.

\subsection*{Erlang}
\label{subsec:State_of_the_art::Candidates::Erlang}
Erlang is a relatively niche programming language originally designed for the use in telephony applications. It features a high focus on concurrency and a garbage collector which is enabled through the execution inside the BEAM virtual machine. Today it is most often used in soft real-time computing~\footnote{see \url{https://en.wikipedia.org/wiki/Real-time_computing}} because of it's error tolerance, hot code reload capabilities and lock-free concurrency support.

- brief history?

- code example (not hello world rather show message passing)

- Upsides
    - Great concurrency
    - Message passing is default (no locks)
    - Hot swap?
- Downsides
    - Bad interfacing to other languages
    - Weird syntax
    - Limited (community/support?)


\subsection*{Go}
\label{subsec:State_of_the_art::Candidates::Go}
Go is a relatively new programming language which focusses on simplicity and clarity while not sacrificing too much performance. Initially developed by Google it aims to ``make it easy to build simple, reliable and efficient software'' (//cite). It is statically typed, offers a garbage collector, basic type inference and a large standard library. Go's syntax is loosely inspired by C but made some major changes like removing the mandatory semicolon at the end of commands and changing the order of types and identifiers. It was chosen as a candidate because it provides simple concurrency primitives as part of the language (so called \textit{goroutines}) while having a familiar syntax and beeing reasonably performant.~\cite{intro_go} It also compiles to native code without external dependencies which makes it usable on cluster computers without many additional libraries installed.

The chosen code example demonstrated two key features which are essential to concurrent programming in Go - the already mentioned goroutines as well as channels used for synchronization purposes. These provide a way to communicate with goroutines via message passing.

\lstinputlisting[caption={Go concurrency example}, label={lst:example.go}, style=go]{code/example.go}

Initially developed for server scenarios Go has seen production use in many different areas. At Google it is used for various internal project such as the download service ``dl.google.com'' which has been completely rewritten from C++ to Go in 2012. The new version can handle more bandwith while using less memory. It is also noteable that the Go codebase is about half the size of the legacy application with increases test coverage.~\cite{go_dl_google}

All these statistics show the core focus of the language // simplicity and avoidance of boilerplate

- brief history

- Prediction implementation
    - A bit of syntax weirdness
    - Relatively quick PoC with decent concurrency aspects
    - Some fixing/optimization afterwards regarding common concurrency errors
    -> More time spent after initial PoC but less than in C


\subsection*{Rust}
\label{subsec:State_of_the_art::Candidates::Rust}
The last candidate discussed in this chapter is Rust. Developed in the open but strongly backed by Mozilla Rust aims to directly compete with C and C++ as a systems language. It focuses on memory safety which is checked and verified at compile without (or with minimal) impact on runtime performance. Rust compiles to native code using a custom fork of the popular LLVM\footnote{\url{http://www.llvm.org}} as backend and is compatible to common tools like the gdb\footnote{\url{http://www.gnu.org/software/gdb/}} debugger which makes integration into existing workflows a bit easier.

Out of the here discussed languages Rust is closest to C while attempting to fix common mistakes made possible by it's loose standard allowing undefined behaviour. (//wording?) Memory safety is enforced through a very sophisticated model of ownership. It is based on common concepts which are already employed on concurrent applications. The basic rule is that every piece of allocated memory is \textit{owned} by \textit{one} entity in the program (typically a variable) and only the owner can change the contents of that memory. To allow more complex algorithms values can be borrowed

- Key features

- Up/Downside

- Prediction Implementation
    - Moderatly quick PoC without concurrency at first
    - Nearly only otimization afterwards since compilation secures memory safety
    -> More time spent before initial PoC than after


\subsection*{Comparison}
\label{subsec:State_of_the_art::Candidates::Comparison}

% will probably not fit
\begin{tabular}{llll}
    \toprule
    % Header
        & Python
        & Erlang
        & Go
        & Rust \\
    \midrule

    Execution model
        & interpreted
        & compiled to bytecode
        & compiled to native code
        & compiled to native code \\

    Advantages
        & low barrier of entry
        & builtin (lockfree) concurrency support
        & adv go
        & adv rust \\

    Disadvantages
        & speed
        & obscure syntax
        & mandatory runtime
        & disadv rust \\

    Relative speed
        & slow to average
        & average to fast
        & speed go
        & speed rust \\
    \bottomrule
\end{tabular}


\chapter{Concept}
\label{ch:Concept}

\abstract{The first section of the third chapter describes the existing application this evaluation is based on. In addition the various phases of the development process are roughly illustrated.
}

\section{Overview of the case study \textit{streets4MPI}}
\label{sec:Concept::Overview}

As stated in~\autoref{sec:Introduction::Goals} the concept for the implementations to compare is inspired by \textit{streets4MPI}, which was implemented to evaluate Python's usefulness for ``computational intensive parallel applications''~\cite[p.3]{streets_report}. It was written by Julian Fietkau and Joachim Nitschke in scope of the module ``Parallel Programming'' in Spring 2012 and makes heavy use of the various libraries of the Python ecosystem. \autoref{fig:architecture_streets4mpi} provides a rough overview about the architecture of \textit{streets4MPI}.

\begin{figure}[htb]
    \centering
    \includegraphics[width=.75\textwidth]{img/architecture_streets4mpi.png}
    \caption{Architecture overview: Streets4MPI~\cite[p. 9]{streets_report}}
    \label{fig:architecture_streets4mpi}
\end{figure}

\textit{streets4MPI} parses \gls{osm} input data, builds a directed graph and repeatedly computes shortest paths for a set amount of ``trips'' (randomly chosen node pairs from the graph) in the graph. Over time it gradually modifies the graph based on results of previous iterations to emulate structural changes in the traffic network in the simulated area. The results can then optionally be written to a custom output format which is visualizable by an additional script~\cite{streets_report}.

As mentioned the applications written in scope of this thesis perform only the graph calculations discarding the produced results. Consequently the implementations do not produce any data besides the runtime statistics externally acquired by benchmark tools.

\section{Differences and limitations}
\label{sec:Concept::Differences}

Although the evaluated implementations are based on the original \textit{streets4MPI}, there are some key implementational differences. This section gives a brief overview over the most important aspects that have been changed and describes both the original application's functionality as well as the derived implementations.

In the remaining part of the thesis the different applications will be referenced quite frequently. For brevity the language implementations to compare will be called by the following scheme: ``streets4<language>''. The Go version for example is called ``streets4go''.

\subsection*{Input format}
\label{subsec:Concept::Differences::Input}

The original \textit{streets4MPI} uses the somewhat dated \gls{osm} \gls{xml} format\fnote{\url{http://wiki.openstreetmap.org/wiki/OSM_XML}} as input which is parsed by \textit{imposm.parser}\fnote{\url{http://imposm.org/docs/imposm.parser/latest/}}. It then builds a directed graph via the \textit{python-graph}\fnote{\url{https://code.google.com/p/python-graph/}} library to base the simulation on~\cite{streets_report}.

The derived versions require the input to be in ``.osm.pbf'' format. This newer version of the \gls{osm} format is based on Google's \gls{protobuf} and is superior to the \gls{xml} variant in both size and speed~\cite{osm_wiki_pbf}. It also simplifies multi language development because the code performing the actual parsing is auto generated from a language independent description file. There are \gls{protobuf} backends for C, Rust and Go which can perform that generation.

\subsection*{Simulation}
\label{subsec:Concept::Differences::Simulation}

The simulation in the base application is based on randomly picked node pairs from the source graph. For these trips the shortest path is calculated by Dijkstra's algorithm as seen in \cite{cormen} and a random factor called ``jam tolerance'' is introduced to avoid oscillation in between iterations~\cite{streets_report}. Then after some time has passed in the simulation, existing streets get expanded or shut down depending on their usage.

The compared implementations of this thesis also perform trip based simulation but without the added randomness and street modification. Also the edge weights are not calculated based on speed limit and length of the street. Instead the derived implementations calculate the length once from the corrdinates of the corresponding nodes and use this as edge weigth directly. The concrete algorithm is a variant of the \shinline{Dijkstra-NoDec} algorithm as seen in~\cite[p. 16]{dijkstra_utcs}. It was mainly chosen because of its reduced complexity in required data structures which again reduces complexity and scope. The algorithm is implemented separately in all three languages so it could theoretically get benchmarked standalone to get clearer results. Mainly because of time constraints this was not attempted in this thesis.

\subsection*{Concurrency}
\label{subsec:Concept::Differences::Concurrency}

\textit{streets4MPI} parallelizes its calculations on multiple processes that communicate via message passing. This is achieved with the aforementioned \textit{MPI4Py} library which delegates to a native \gls{mpi} implementation installed on the system. If no supported implementation is found it falls back to a pure Python solution but the native one should be preferred for maximum performance.

Although Rust as well as Go can integrate decently with existing native code, the reimplementations will be limited to shared memory parallelization on threads. This was mostly decided to evaluate and compare the language inherent concurrency constructs rather than the quality of their foreign funtion interfaces. To achieve a fair comparison \textit{streets4c} will use \textit{OpenMP}~\fnote{\url{http://www.openmp.org}} as it is the de facto standard for simple thread parallelization in C. Of course this solution might not match the performance of hand optimized implementations parallelized with the help of \textit{pthreads} but since the focus is on simple concurrency in the context of scientific applications \textit{OpenMP} was selected as the framework of choice.

\section{Implementation process}
\label{sec:Concept::Implementation}

The implementation process was performed iteratively. Certain milestones were defined and implemented in all three languages. The process only advanced to the next phase when the previous milestone was reached in all applications. This approach was chosen to allow for a fair comparison of the different phases of development. If the implementations would have been developed one after another to completion (or in any other arbitrary order), this might have introduced a certain bias to the evaluation because of possible knowledge about the problem aquired in a previous language translating to faster results in the next one.

\begin{figure}[htb]
    \centering
    \includegraphics[width=\textwidth]{img/milestone_timeline.png}
    \caption{Milestone overview}
    \label{fig:timeline}
\end{figure}

For each phase various characteristics were captured and compared to highlight the languages' features and performance in the various areas. While the main development and test runs were performed on a laptop the final application was run on a high performance machine provided by the research group Scientific Computing to compare scalability beyond common desktop level processors.

\setcounter{subsection}{-1}

\subsection{Setting up the project}
\label{subsec:Concept::Implementation::Setup}

The first phase of development was to create project skeletons and infrastructure for the future development. The milestone was to have a working environment in place where the sample application could be built and executed. While this is certainly not the most important or even interesting part it did show the differences in comfort between the various toolchains.

\subsection{Counting nodes, ways and relations in an .osm.pbf file}
\label{subsec:Concept::Implementation::Counting}

The first real milestone was to read a .osm.pbf file and count all nodes, ways and relations in it. This was done to get familiar with the required libraries and the file format in general. The time recorded began from the initial project created in phase 0 and finished after the milestone was reached. As this is the most input and output intensive phase it already showed some key differences between the candidates both in speed as well as memory consumption.

\subsection{Building a basic graph representation}
\label{subsec:Concept::Implementation::Graph_Representation}

The next goal was to conceptionally build the graph and related structures the simulation would later operate on. This involved thinking about the relation between edges and nodes as well as the choice of various containers to store the objects efficiently while also keeping access simple. This milestone tested the language's standard libraries and expressiveness in terms of typed containers.

\subsection{Verifying structure and algorithm}
\label{subsec:Concept::Implementation::Verification}

After the base structure to represent graphs and calculate shortest paths was in place it was time to validate the implementations. Unfortunately the OSM data used in the first phase contained too much nodes and ways to be able to efficiently verify any computed results. Therefore a small example graph was manually populated and fed to the algorithm.

\subsection{Benchmarking graph performance}
\label{subsec:Concept::Implementation::SequentialBenchmark}

The fourth milestone was preliminary benchmark of the implementations. The basic idea was to parse the \gls{osm} data used in phase one and build the representing graph. After that the shortest path algorithm is executed once for each node. The total execution time as well as the time taken for each step (building the graph and calculating shortest paths) should be measured and compared as well as the usual memory statistics from previous phases.

\subsection{Benchmarking parallel execution}
\label{subsec:Concept::Implementation::ParallelBenchmark}

The fifth and final phase consisted of modifying the existing benchmark to operate in parallel via threading and benchmarking the results for various configurations. While all the development and previous benchmarks were performed on a personal laptop the final benchmarks were taken on a computation node of the research group to gather relevant results in high concurrency situations.

\subsection{Cluster preperation}
\label{subsec:Concept::Implementation::ClusterPreparation}

\section{Overview of evaluated criteria}
\label{sec:Concept::Criteria}

\begin{itemize}
    \item Performance
    \begin{itemize}
        \item Execution Time
        \item Memory Footprint (consumption total + allocation counts)
    \end{itemize}
    \item Productivity
    \begin{itemize}
        \item \acrshort{sloc} Count
        \item Development Time
        \item Tooling Support
        \item Library Ecosystem
        \item Parallelization Effort
    \end{itemize}
\end{itemize}

\section{Related work}
\label{sec:Concept::Related}

The search for new programming languages which are fit for \gls{hpc} is not a recently developing trend. There have been multiple studies and evaluations but so far none of the proposed languages have gained enough traction to receive widespread adoption. Also most reports focused on the execution performance without really considering additional software metrics or developer productivity~\cite{related_multicore}. at least adds lines of code and development time to the equation but both of these metrics only allow for superficial conclusions about the code quality.

From the candidates presented here Go in particular has been compared to traditional \gls{hpc} languages with mixed results. Although its regular execution speed is somewhat lacking \cite{related_sor_study} showed the highest speedup from parallelization amongst the evaluated languages which is very promising considering high concurrency scenarios like cluster computing. Rust on the other hand has not been seriously evaluated in the \gls{hpc} context probably due to it still being developed.


\chapter{Implementation}
\label{chap:Implementation}

\textit{%
In diesem Kapitel ...
}
\bigskip


Milestones:
    - Count all nodes, ways and relations in hamburg-latest.osm.pbf
        - Rust:
            - SLOC: 37
            - dev-time: 1989 -> 33:09 min
            - run-time:
                - -O0: target/streets4rust ../osm/hamburg-latest.osm.pbf  27,61s user 0,12s system 99\% cpu 27,749 total
                - -O3: target/release/streets4rust ../osm/hamburg-latest.osm.pbf  2,59s user 0,13s system 99\% cpu 2,722 total
            - allocs: total heap usage: 11,373,558 allocs, 11,373,557 frees, 2,186,107,072 bytes allocated
            - counts: Found 2180418 nodes, 409424 ways and 7182 relations in ../osm/hamburg-latest.osm.pbf
            - notes:
                - easy dependency management, huuge optimization gains
        - Go:
            - SLOC: 55 (+ helper functions)
            - dev-time: 1276 -> 21:16 min
            - run-time:
                - GOMAXPROCS=1: ./streets4go ../osm/hamburg-latest.osm.pbf  4,85s user 0,05s system 101\% cpu 4,846 total
                - GOMAXPROCS=8: ./streets4go ../osm/hamburg-latest.osm.pbf  9,00s user 0,28s system 672\% cpu 1,381 total
            - allocs: total heap usage: 11,164,068 allocs, 11,000,199 frees, 1,447,543,184 bytes allocated
            - counts: Found 2180418 nodes, 409424 ways and 7182 relations in ../osm/hamburg-latest.osm.pbf
            - notes:
                - library parallelizeable, but singlethreaded slower
        - C:
            - SLOC: 55
            - dev-time: 3078 -> 51:18 min
            - run-time:
                - -O0: ./streets4c ../osm/hamburg-latest.osm.pbf  0,95s user 0,07s system 99\% cpu 1,017 total
                - -O3: ./streets4c ../osm/hamburg-latest.osm.pbf  0,92s user 0,08s system 99\% cpu 0,994 total
            - allocs: total heap usage: 2,390,566 allocs, 2,390,566 frees, 372,758,206 bytes allocated
            - counts: Found 2180418 nodes, 409424 ways and 7182 relations in ../osm/hamburg-latest.osm.pbf
            - notes:
                - no library available, linking problems, freeing problems, fastest solution (run), slowest solution (dev)
    - Write basic graph structure
        - C:
            - SLOC: 385 (*.c), 136 (*.h) -> 521 total
            - dev-time: 12110 - 3078 = 9032 -> 02:30:32
            - notes:
                - glib verbosity, missing generics / typesafety
        - Rust:
            - SLOC: 171 total
            - dev-time: 6457 - 1989 = 4468 -> 01:14:28
            - notes:
                - lifetime gotchas (explicit readonly of dg.g etc.)
        - Go:
            - SLOC: 196 total (including util (priorityqueue) but excluding test)
            - dev-time: 5242 - 1276 = 3966 -> 01:06:06
            - notes:
                - testability


\chapter{Evaluation}
\label{ch:Evaluation}

\abstract{This chapter provides the analysis of the statistics gathered from the final implementations. As stated in the introduction the evaluation considers raw performance characteristics as well as developer productivity. Both areas are evaulated based on statistics from the previous chapter.
}

Preface: All data in this chapter was gathered from a high performance computer by courtesy of the research group Scientific Computing. The machine has access to four 12-core processors and 128 GB of memory. It is therefore ideal to compare shared memory performance on a large scale.

\section{Performance}
\label{sec:Evaluaton::Performance}

In \acrlong{hpc} the most important criteria when evaluating a language is performance. The important statistic that was tracked to compare performance is execution time. The benchmarks that were performed on the development laptop also roughly measured memory usage but that proved difficult to automate on the remote machine. It is therefore not directly included in this final evaluation. \autoref{tb:final_execution_time} shows the benchmark results in varying concurrency scenarios from singlethreaded execution up to 48 calculating in parallel.

\begin{table}[htb]
    \centering
    \begin{tabular}{rccc}
        \toprule
        % Header
            threads/goroutines
            & C
            & Go
            & Rust \\
        \midrule

            1
            & 21:51:18
            & 16:48:19
            & 14:15:06 \\

            2
            & 12:29:56
            & 10:21:36
            & 09:12:47 \\

            4
            & 07:16:34
            & 05:58:35
            & 05:09:56 \\

            8
            & 04:13:04
            & 03:01:54
            & 02:49:35 \\

            12
            & 03:17:28
            & 02:06:08
            & 01:55:33 \\

            24
            & 02:06:08
            & 01:13:47
            & 01:03:34 \\

            48
            & 01:21:58
            & 00:53:54
            & 00:44:54 \\

        \bottomrule
    \end{tabular}
    \caption{Execution time of the final applications (100K nodes)}
    \label{tb:final_execution_time}
\end{table}

These results already contain the first real surprise. C that was chosen as a comparitive baseline, since it is one of the two big programming languages in \gls{hpc}, is the slowest of the three compared languages in all configurations. In contrast the preliminary benchmarks on the development laptop showed C while not ahead at least on second place in the performance comparison. As briefly mentioned in the \hyperref[subsec:Implementation::ClusterPreparation::C]{previous chapter} this performance regression might have been caused by the two unoptimized libraries that were compiled on the development laptop and copied to the cluster. However this shows that C is still very much compiler and machine dependent.

In contrast the Go binary that was also compiled on the development laptop was executed without any changes on the target machine and shows great result even reaching similar performance to Rust in the high concurrency configurations. This shows that a garbage collected language is not immediately unsuitable for use in \gls{hpc}. Combined with the portability caused by full static linking Go might very well be suited for cluster computations on nodes with a minimum of system libraries installed.

Finally Rust demonstrates that it might be a competent successor to C in \gls{hpc}. It is the fastest language out of the compared three across all scenarios while provding additional memory safety through its unique typesystem. It prevented multiple errors from compiling in the Rust implementation throughout the whole development process. On one occasion it even revealed an error which had gone unnoticed in both C and Go. This really highlights how static analysis can provide safety without sacrificing performance.

Another important statistic to compare is the parallel speedup. A slow execution time alone does not mean a language is completely unfit for \gls{hpc} because the implementation might simply be flawed to begin with. If this is the case the application can still offer above average speedups making it viable for high concurrency scenarios. \autoref{tb:final_speedup} lists the achieved speedup for each language in the same configurations as above.

\begin{table}[htb]
    \centering
    \begin{tabular}{rccc}
        \toprule
        % Header
            threads/goroutines
            & C
            & Go
            & Rust \\
        \midrule

            1
            & \hspace{6pt}1.0000
            & \hspace{6pt}1.0000
            & \hspace{6pt}1.0000 \\

            2
            & \hspace{6pt}1.7486
            & \hspace{6pt}1.6221
            & \hspace{6pt}1.5469 \\

            4
            & \hspace{6pt}3.0037
            & \hspace{6pt}2.8119
            & \hspace{6pt}2.7590 \\

            8
            & \hspace{6pt}5.1816
            & \hspace{6pt}5.5432
            & \hspace{6pt}5.0424 \\

            12
            & \hspace{6pt}6.6406
            & \hspace{6pt}7.9941
            & \hspace{6pt}7.4003 \\

            24
            & 10.3961
            & 13.6659
            & 13.4520 \\

            48
            & 15.9980
            & 18.7072
            & 19.0445 \\

        \bottomrule
    \end{tabular}
    \caption{Parallel speedup of the final applications (100K nodes)}
    \label{tb:final_speedup}
\end{table}

Again the results are interesting for multiple reasons. C scales very well up to four threads but falls off quite heavily. Rust and Go are evenly matched with Go scaling better up to the final benchmark with 48 threads where it gets beaten slightly by Rust. It would be interesting to see the trend continue here but unfotunately the target machine ``only'' offered 48 logical cores. C' strong scaling in the lower thread counts shows OpenMP's efficiency in generating threaded code for common desktop scenarios. In the high concurrency configurations though the scaling diminishes resulting in a big discrepancy of about 3 (\textasciitilde13\%). For these use cases it might be worthwhile to implement custom parallelization with \textit{pthreads}.\fnote{\url{http://pubs.opengroup.org/onlinepubs/9699919799/basedefs/pthread.h.html}} This approach will most likely result in much higher development time but might yield better performance results for higher thread counts.

Rust and Go scale both comparably well. Although the final speedup of about 19 is not great for 48 threads it is still a serious improvement about the serial version. It is also important to note that the threads share the work with statically which means there is no load balancing once the threads are started. This results in some threads exiting early effectively reducing the speedup. To solve this the work could be dynamically distributed for example through a queue like construct which threads use to retrieve new tasks. A possible implementation could be channel based as both Go and Rust offer those as part of the standard library.

\begin{figure}[htb]
    \centering
    \makebox[\textwidth][c]{%
        \subfloat[Execution time]{%
        \begin{tikzpicture}
            \begin{axis}[
                xlabel={Amount of threads},
                ylabel={Execution time (sec)},
                xtick={1,2,4,8,12,24,48},
                xmode=log,
                log base x=2,
                log ticks with fixed point,
                scaled y ticks = false,
                ylabel near ticks,
                legend pos=north east]

                \addplot[red] plot coordinates {
                    (1,78678)
                    (2,44996)
                    (4,26194)
                    (8,15184)
                    (12,11848)
                    (24,7568)
                    (48,4918)};
                \addlegendentry{C}

                \addplot[color=blue] plot coordinates {
                    (1,60499)
                    (2,37296)
                    (4,21512)
                    (8,10914)
                    (12,7568)
                    (24,4427)
                    (48,3234)};
                \addlegendentry{Go}

                \addplot[color=green] plot coordinates {
                    (1,51306)
                    (2,33167)
                    (4,18596)
                    (8,10175)
                    (12,6933)
                    (24,3814)
                    (48,2694)};
                \addlegendentry{Rust}

            \end{axis}
        \end{tikzpicture}
        }
        \subfloat[Parallel speedup]{%
        \begin{tikzpicture}
            \begin{axis}[
                xlabel={Amount of threads},
                ylabel={Speedup},
                xtick={1,2,4,8,12,24,48},
                xmode=log,
                ymode=log,
                log base x=2,
                log base y=2,
                log ticks with fixed point,
                legend pos=north west]

                \addplot[red] plot coordinates {
                    (1,1.0000)
                    (2,1.7486)
                    (4,3.0037)
                    (8,5.1816)
                    (12,6.6406)
                    (24,10.3961)
                    (48,15.9980)};
                \addlegendentry{C}

                \addplot[color=blue] plot coordinates {
                    (1,1.0000)
                    (2,1.6221)
                    (4,2.8119)
                    (8,5.5432)
                    (12,7.9941)
                    (24,13.6659)
                    (48,18.7072)};
                \addlegendentry{Go}

                \addplot[color=green] plot coordinates {
                    (1,1.0000)
                    (2,1.5469)
                    (4,2.7590)
                    (8,5.0424)
                    (12,7.4003)
                    (24,13.4520)
                    (48,19.0445)};
                \addlegendentry{Rust}

                \addplot[color=black] plot coordinates {
                    (1,1.0000)
                    (2,2.0000)
                    (4,4.0000)
                    (8,8.0000)
                    (12,12.0000)
                    (24,24.0000)
                    (48,48.0000)};
                \addlegendentry{ideal}

            \end{axis}
        \end{tikzpicture}
        }
    }
    \caption{Performance metrics across the various milestones}
    \label{fig:performance_metrics}
\end{figure}

\autoref{fig:performance_metrics} compares the execution time and speedup side by side. //continue this..

\section{Additional metrics / productivity}
\label{sec:Evaluation::Metrics}

Next to performance the second main area evaluated in this thesis is developer productivity. The two criteria tracked to compare this category are \gls{sloc} count and development time. While the \gls{sloc} count is certainly not the best code quality metric it allows for some basic conclusions. Less lines can contain potentially less errors, lowering maintenance costs, and \textit{should} take less time develop. This is obviously not always the case but the following graphs mostly confirm this correlation for the evaluated implementations.

\begin{figure}[htb]
    \centering
    \makebox[\textwidth][c]{%
        \subfloat[Development time]{%
        \begin{tikzpicture}
            \begin{axis}[
                xlabel={Milestone},
                ylabel={Total time (sec)},
                xtick={1,2,3,4,5,6},
                scaled y ticks = false,
                ylabel near ticks,
                legend pos=north west]

                \addplot[red] plot coordinates {
                  (1,3078)
                  (2,12110)
                  (3,18920)
                  (4,23392)
                  (5,23883)};
                \addlegendentry{C}

                \addplot[color=blue] plot coordinates {
                  (1,1276)
                  (2,5242)
                  (3,9851)
                  (4,13227)
                  (5,13703)};
                \addlegendentry{Go}

                \addplot[color=green] plot coordinates {
                  (1,1989)
                  (2,6457)
                  (3,10335)
                  (4,13055)
                  (5,14698)};
                \addlegendentry{Rust}

            \end{axis}
        \end{tikzpicture}
        }
        \subfloat[SLOC counts]{%
        \begin{tikzpicture}
            \begin{axis}[
                xlabel={Milestone},
                ylabel={SLOC},
                xtick={1,2,3,4,5,6},
                legend pos=north west]

                \addplot[red] plot coordinates {
                  (1,163)
                  (2,385)
                  (3,494)
                  (4,757)
                  (5,777)
                  (6,668)};
                \addlegendentry{C}

                \addplot[color=blue] plot coordinates {
                  (1,55)
                  (2,196)
                  (3,275)
                  (4,359)
                  (5,381)
                  (6,285)};
                \addlegendentry{Go}

                \addplot[color=green] plot coordinates {
                  (1,36)
                  (2,170)
                  (3,232)
                  (4,292)
                  (5,314)
                  (6,253)};
                \addlegendentry{Rust}

            \end{axis}
        \end{tikzpicture}
        }
    }
    \caption{Productivity metrics across the various milestones}
    \label{fig:productivity_metrics}
\end{figure}

The sixth milestone in the \gls{sloc} count graph references the final stripped down versions as introduced in \autoref{sec:Implementation::ParallelBenchmark}. This small phase was not included in the development time comparison as this statistic was not tracked.

Comparing developer productivity C comes in last by a large margin. The high development time can be traced back to the manual implementation of common data structures and the high amount of memory and type related errors encountered during the development. This naturally lead to a much higher \gls{sloc} count which was increased further by code duplication caused by the conventional header files. It is important to note here that the \gls{sloc} count for C is also partially caused by the dominant bracing style. While Go forces the programmer to set all curly braces on the same line as the preceeding statement the C implementations were written in a different style always placing curly braces on new lines. Rust follows the Go rules but only encourages them as a convention allowing for other styles to be used. This in turn produces some extra lines on the C side while Rust and Go remain unaffected. However the difference is still way too significant to be only attributed to the style choice.

Rust and Go on the other hand both allow for substantial productivity gains. There seems to be just a little tradeoff between the two tracked criteria with Rust requiring slightly less lines of code while Go is a little faster to develop in. Although the results are so close together that it does not matter very much when compared to C. While these changes are certainly largely caused by the languages themselves especially the lower development time is also possible through the superior tooling. Both Go and Rust offer excellent tool support for dependency management and other parts of the build process. Especially for compilations on foreign machines this is invaluable because the application is mostly independent from systemwide installed libraries.

All these results reinforce the inital motivation from the \hyperref[sec:Introduction::Motivation]{Introduction} for a new successor to C in the context of \acrlong{hpc}. Considering productivity only both Go and Rust offer excellent advantages over C.


\chapter{Conclusion}
\label{ch:Conclusion}

\abstract{In this final chapter a short summary is given and the thesis concludes with a brief description of possible improvements and future work.}

\section{Summary}
\label{sec:Conclusion::Summary}

\section{Improvements and future work}
\label{sec:Conclusion::Improvements}

Although this evaluation already yielded some interesting results regarding new programming languages in \gls{hpc} there are still lots of potential research topics in this area. Also there are some missed opportunities for a more complete result which were just not considered or skipped due to time constraints. This section addresses both of these areas and concludes this thesis.

\subsection*{General code quality}
\label{subsec:Conclusion::Improvements::CodeQuality}

As mentioned a few times in the thesis the quality of the implementations might not be optimal especially in the case \textit{streets4C}. This can mostly be attributed to the author's lack of equal experience in the three evaluated languages. While this is to a point intentional, as scientists might not be proficient in these languages either the implementations could be reviewed by language experts to really demonstrate the absolute highest possible performance.

\subsection*{Limited benchmark configurations}
\label{subsec:Conclusion::Improvements::Configuration}

Although the benchmark results presented in the previous chapter allow for some decent conclusions the configurations could have been extended. Especially the problem size which was fixed at 100K nodes could have been varied for multiple thread amounts. Another possible comparison could include different compilers for C and Go (Rust currently only has one reference implementation) or multiple compiler and library versions revealing possible regressions across versions.

\subsection*{Limitation to shared memory}
\label{subsec:Conclusion::Improvements::SharedMemory}

While thread based concurrency is certainly an important aspect in \gls{hpc} the dominant model is distributed memory communicating via message passing. This technique was not evaluated in this thesis because of missing library support but the general performance of the languages should still be applicable. As both Rust and Go have good capabilities to link to native C libraries it might be possible to use a standard \gls{mpi} implementation today. A complete library written in the target language should be preferred whenever available though because interfacing with C often restricts the types which can be used. When these libraries have matured enough (if ever) it might be very valuable to reassess the candidates in the \gls{hpc} context.


% Bibliography
% \nocite{*}
\printbibliography

% Lists of *
\microtypesetup{protrusion=false} % disable protrusion locally for the following lists

\listoffigures
\listoftables
\lstlistoflistings

\microtypesetup{protrusion=true}  % reenable protrusion

% Appendices
\begin{appendices}
% \chapter{Glossary}
\glsaddall
\printglossaries

\chapter{System configuration}
\label{ap:Configuration}

\section*{Development laptop}
\label{ap:Configuration::Laptop}

\lstinputlisting[caption={Output of \mdinline{uname -a}}, label={lst:uname.cfg}, style=plain]{misc/uname.cfg}

\lstinputlisting[caption={Output of \mdinline{lscpu}}, label={lst:lscpu.cfg}, style=plain]{misc/lscpu.cfg}

\section*{Cluster}
\label{ap:Configuration::Cluster}

\lstinputlisting[caption={Output of \mdinline{uname -a}}, label={lst:cl_uname.cfg}, style=plain]{misc/cl_uname.cfg}

\lstinputlisting[caption={Output of \mdinline{lscpu}}, label={lst:cl_lscpu.cfg}, style=plain]{misc/cl_lscpu.cfg}

\chapter{Software versions}
\label{ap:Versions}

These are the compiler and toolchain versions which were used to develop and compile all code in this thesis.
\\ \\
\lstinputlisting[caption={Output of \mdinline{gcc -{}-version}}, label={lst:gcc.version}, style=plain]{misc/gcc.version}

\lstinputlisting[caption={Output of \mdinline{go version}}, label={lst:go.version}, style=plain]{misc/go.version}

\lstinputlisting[caption={Output of \mdinline{rustc -{}-version}}, label={lst:rustc.version}, style=plain]{misc/rustc.version}

\lstinputlisting[caption={Output of \mdinline{cargo -{}-version}}, label={lst:cargo.version}, style=plain]{misc/cargo.version}

\chapter{Final notes}
\label{ap:notes}

All source code is available at \url{https://github.com/mrfloya/thesis-ba} under various \gls{osi} approved licenses. The versions containing all code from intermediate milestones are in a separate branch called \shinline{incremental}. The \shinline{master} branch only consists of the final variants that were used in the benchmarks on the cluster.

The Rust implementation was compiled with the version shown in \autoref{lst:rustc.version}. Unfortunately it as not possible to compile the code with the current beta version at the time of this writing (1.0.0-beta.2). Because of this the application will most likely not compile with 1.0.0 either. The code in the repository mentioned above will be updated as soon it compiles with a stable Rust release.

\end{appendices}

\newpage

\thispagestyle{empty}

\chapter*{}

% Eidesstaatliche Erklärung
\section*{Erklärung}

Ich versichere, dass ich die Arbeit selbstständig verfasst und keine anderen, als die angegebenen Hilfsmittel -- insbesondere keine im Quellenverzeichnis nicht benannten Internetquellen -- benutzt habe, die Arbeit vorher nicht in einem anderen Prüfungsverfahren eingereicht habe und die eingereichte schriftliche Fassung der auf dem elektronischen Speichermedium entspricht.

\smallskip

\textbf{Optional:} Ich bin mit der Einstellung der Bachelor-Arbeit in den Bestand der Bibliothek des Fachbereichs Informatik einverstanden.

\bigskip
\bigskip
\bigskip

Hamburg, den \leadingzero{\day}.\leadingzero{\month}.\leadingzero{\year} \quad \dotfill

\end{document}
